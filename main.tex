\documentclass[french]{ega}

\RenewDocumentCommand\chaptername{}{\normalfont\textsc{chapitre}}

\usepackage{multicol}
\usepackage{xcolor}
\usepackage{biblatex}
\usepackage[chapters]{dieudonne}

\setcounter{tocdepth}{20}

\newtheorem{theorem}[node]{Theorem}

\addbibresource{refs.bib}


\title{The \emph{Dieudonné} \LaTeX{} package}
\author{Jonathan Sterling}


\begin{document}

\maketitle

\tableofcontents*

\mainmatter
\CounterZeroNext{chapter}
\chapter{Préliminaires}

\section{Anneaux de Fractions}

\CounterZeroNext{subsection}
\subsection{Anneaux et algèbres}

\vspace{-1em}
\begin{minipage}[t]{0.5\textwidth}
  \begin{node}
    %
    Tout les anneaux considéré dans ce Traité posséderont un
    \emph{élément unité}; tout le modules sur un tel anneau seront
    supposés \emph{unitaires}; le homomorphismes d'anneaux seront toujours
    supposés \emph{transformer l'élément unité en élément unité};
    sauf mention expresse du contraire, un sous-anneau d'un anneau \(A\) sera
    supposé \emph{contenir l'élément unité de \(A\)}. Nous considérerons
    surtout des anneaux \emph{commutatifs}, et lorsque nous parlerons  d'anneau
    sans préciser, il sera sous-entendu qu'il s'agit d'un anneau commutatif. Si
    \(A\) est un anneau non nécessairement commutatif, par \(A\)-module nous
    entendrons toujours un module \emph{à gauche}, sauf mention expresse du
    contraire.
    %
  \end{node}
\end{minipage}
\quad
\begin{minipage}[t]{0.5\textwidth}
  \addtocounter{node}{-1}

  \begin{node}
    %
    All the rings considered in this treatise possess a \emph{unit element};
    all the modules on such a ring are assumed to be unitary; the homomorphisms
    of rings will be assumed to \emph{take the unit element to the unit element};
    unless specifically mentioned to the countrary, a subring of a ring \(A\)
    will be assumed to \emph{contain the unit element of \(A\)}. We consider
    mostly \emph{commutative} rings, and when we speak of a ring without
    specifying, it will be implied that  it is a commutative ring. If \(A\) is a
    not necessarily commutative ring, by \(A\)-module we always mean a
    \emph{left} module, unless mentioned specifically to the contrary.
    %
  \end{node}

\end{minipage}
\medskip



\textcolor{gray}{[omitted]}

\setcounter{subsection}{2}
\subsection{Propriétés fonctorielles}

\begin{node}\label{node:foo}
  %
  Soient \(M, N\) deux \(A\)-modules, \(u\) un \(A\)-homomorphism \(M\to N\). Si \(S\) est
  une partie multiplicative de \(A\), on définit un \(S^{-1}A\)-homorphisme
  \(S^{-1}M\to S^{-1}N\), noté \(S^{-1}u\), en posan \((S^{-1})(m/s) = u(m)/s\); si
  \(S^{-1}M\) et \(S^{-1}N\) sont canoniquement identifiés à
  \(S^{-1}\otimes_{A}M\) et \(S^{-1}\otimes_{A}N\) (1.2\textperiodcentered5),
  \(S^{-1}u\) est identifié à \(1\otimes u\). Se \(I\) est un troisième
  \(A\)-module, \(v\) un \(A\)-homomorphisme \(N\to P\), on a \(S^{-1}(v\circ u) =
  (S^{-1}v)\circ (S^{-1}u)\); autrement dit \(S^{-1}M\) est on \emph{foncteur
  covariant en \(M\)}, de le catégorie des \(A\)-modules dans celle des
  \(S^{-1}A\)-modules (\(A\) et \(S\) étant fixés).
\end{node}

\begin{node}
  Le foncteur \(S^{-1}M\) est \emph{exact}; autrement dit, si la suite
  \begin{equation}
    M\xrightarrow{u} N\xrightarrow{v} P
  \end{equation}
  est exacte, il en est de même de la suite
  \begin{equation}
    S^{-1}M\xrightarrow{S^{-1}u}S^{-1}N\xrightarrow{S^-1v}S^{-1}P.
  \end{equation}

  En particulier, si \(u:M\to N\) est injectif (resp.\ surjectif), il en est de
  même de \(S^{-1}u\); si \(N\) et \(P\) sont deux sous-modules de \(M\), \(S^{-1}N\) et
  \(S^{-1}P\) s'identifient canoniquement à des sous-modules de \(S^{-1}\), et
  l'on a
  \begin{equation}\label{eq:foo}
    S^{-1}(N+P) = S^{-1}N+S^{-1}P\quad\text{et}\quad S^{-1}(N\cap P) = (S^{-1}N)\cap(S^{-1}P).
  \end{equation}
\end{node}

\begin{node}
  Testing refs: \cref{node:foo} and \cref{eq:foo}
\end{node}


\backmatter

\nocite{*}
\printbibliography

\end{document}
