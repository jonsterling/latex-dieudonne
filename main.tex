\documentclass[oneside,10pt,french]{memoir}
\usepackage{xcolor}
\usepackage{biblatex}
\addbibresource{refs.bib}

\linespread{1.1}\selectfont

\usepackage{amsthm}

\setlrmarginsandblock{1.33in}{*}{1}
\setulmarginsandblock{1.4in}{*}{1}
\setheaderspaces{*}{2\onelineskip}{*}
\checkandfixthelayout

\usepackage[memoir]{dieudonne}

\usepackage[full]{textcomp}
\usepackage{frenchmath}

\usepackage[osfI]{garamondx} 
\usepackage[varqu,varl]{zi4}
\usepackage[garamondx,cmbraces]{newtxmath}

\setlength\cftsectionindent{10pt}
\setlength\cftsubsectionindent{10pt}
\setlength\cftsectionnumwidth{2em}
\setlength\cftsubsectionnumwidth{2em}

\newtheorem{theorem}[node]{Theorem}

\pretitle{\begin{center}\Huge\bfseries}
\posttitle{\par\end{center}\vskip 0.5em}
\predate{\begin{center}\large}
\postdate{\par\end{center}}



\renewcommand{\chaptername}{\normalfont\textsc{chapitre}}
\renewcommand{\chapnumfont}{\normalfont}

\RenewDocumentCommand\chapnamefont{}{\centering\LARGE\scshape\MakeLowercase}
\RenewDocumentCommand\chapnumfont{}{\centering\LARGE\scshape}
\RenewDocumentCommand\chaptitlefont{}{\centering\LARGE\bfseries}
\RenewDocumentCommand\printchaptertitle{m}{\chaptitlefont\MakeUppercase{#1}}

\setsecnumdepth{subsubsection}
\setsecnumformat{\csname jms#1\endcsname}
\NewDocumentCommand\jmssection{}{\S~\thesection.\ \ }
\NewDocumentCommand\jmssubsection{}{\thesubsection.\ \ }
\NewDocumentCommand\jmssubsubsection{}{\thesubsubsection.\ \ }
\setsecheadstyle{\centering\normalsize\bfseries\MakeUppercase}
\setsubsecheadstyle{\noindent\normalfont\bfseries}


\title{The \emph{Dieudonn\'e} \LaTeX{} package}
\author{Jonathan Sterling}


\begin{document}

\maketitle

\tableofcontents*

\mainmatter
\CounterZeroNext{chapter}
\chapter{Pr\'eliminaires}

\setcounter{secnumdepth}{30}

\section{Anneaux de Fractions}

\CounterZeroNext{subsection}
\subsection{Anneaux et alg\`ebres}

\begin{node}
  %
  Tout les anneaux consid\'er\'e dans ce Trait\'e poss\'ederon un
  \emph{\'el\'ement unit\'e}; tout le modules sur un tel anneau seront
  suppos\'es \emph{unitaires}; le homomorphismes d'anneaux seront toujours
  suppos\'es \emph{transformer l'\'el\'ement unit\'e en \'el\'ement unit\'e};
  sauf mention expresse du contraire, un sous-anneau d'un anneau $A$ sera
  suppos\'e \emph{contenir l'\'el\'ement unit\'e de $A$}. Nous consid\'ererons
  surtout des anneaux \emph{commutatifs}, et lorsque nou parlerons  d'anneau
  san pr\'eciser, il sera sous-entendu qu'il s'agit d'un anneau commtatif. Si
  $A$ est un anneau non n\'ecessairement commutatif, par $A$-module nous
  entendron toujours un module \emph{\`a gauche}, sauc mention expresse du
  contraire.
  %
\end{node}


\textcolor{gray}{[omitted]}

\setcounter{subsection}{2}
\subsection{Propri\'et\'es fonctorielles}

\begin{node}
  %
  Soient $M, N$ deux $A$-modules, $u$ un $A$-homomorphism $M\to N$. Si $S$ est
  une partie multiplicative de $A$, on d\'efinit un $S^{-1}A$-homorphisme
  $S^{-1}M\to S^{-1}N$, not\'e $S^{-1}u$, en posan $(S^{-1})(m/s) = u(m)/s$; si
  $S^{-1}M$ et $S^{-1}N$ sont canoniquement identifi\'es \`a
  $S^{-1}\otimes_{A}M$ et $S^{-1}\otimes_{A}N$ (1.2\textperiodcentered5),
  $S^{-1}u$ est identifi\'e \`a $1\otimes u$. Se $I$ est un troisi\`eme
  $A$-module, $v$ un $A$-homomorphisme $N\to P$, on a $S^{-1}(v\circ u) =
  (S^{-1}v)\circ (S^{-1}u)$; autrement dit $S^{-1}M$ est on \emph{foncteur
  covariant en $M$}, de le cat\'egorie des $A$-modules dans celle des
  $S^{-1}A$-modules ($A$ et $S$ \'etant fix\'es).
\end{node}

\begin{node}
  Le foncteur $S^{-1}M$ est \emph{exact}; autrement dit, si la suite
  \begin{equation}
    M\xrightarrow{u} N\xrightarrow{v} P
  \end{equation}
  est exacte, il en est de m\^eme de la suite
  \begin{equation}
    S^{-1}M\xrightarrow{S^{-1}u}S^{-1}N\xrightarrow{S^-1v}S^{-1}P.
  \end{equation}

  En particulier, si $u:M\to N$ est injectif (resp.\ surjectif), il en est de
  m\^eme de $S^{-1}u$; si $N$ et $P$ sont deux sou-modules de $M$, $S^{-1}N$ et
  $S^{-1}P$ s'identifient canoniquement \`a des sous-modules de $S^{-1}$, et
  l'on a
  \begin{equation}
    S^{-1}(N+P) = S^{-1}N+S^{-1}P\quad\text{et}\quad S^{-1}(N\cap P) = (S^{-1}N)\cap(S^{-1}P).
  \end{equation}

\end{node}

\nocite{*}
\printbibliography

\end{document}
